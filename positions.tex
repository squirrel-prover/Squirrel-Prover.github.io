\documentclass[10pt,a4paper]{article}
\usepackage[english]{babel}
\usepackage[utf8]{inputenc}
\usepackage[T1]{fontenc}
\usepackage{lmodern}
\usepackage{url}
\usepackage{xcolor}
\usepackage{xspace}
\usepackage{geometry}
\usepackage{eurosym}
%\usepackage{hyperref}

\geometry{hmargin=2.5cm,vmargin=2.5cm}
\colorlet{darkgreen}{black!40!green}

\usepackage[colorlinks=true, urlcolor=black, citecolor=darkgreen]{hyperref}
\usepackage{microtype}


\newcommand{\ahrefurl}[1]{\url{#1}}
\newcommand{\email}[1]{\href{mailto:#1}{\nolinkurl{#1}}}
%\usepackage[english,frenchb]{babel}

%% to make the bibliography appear as a \subsection*
\makeatletter
\renewenvironment{thebibliography}[1]
     {\subsection*{\refname}%
      \@mkboth{\MakeUppercase\refname}{\MakeUppercase\refname}%
      \list{\@biblabel{\@arabic\c@enumiv}}%
           {\settowidth\labelwidth{\@biblabel{#1}}%
            \leftmargin\labelwidth
            \advance\leftmargin\labelsep
            \@openbib@code
            \usecounter{enumiv}%
            \let\p@enumiv\@empty
            \renewcommand\theenumiv{\@arabic\c@enumiv}}%
      \sloppy
      \clubpenalty4000
      \@clubpenalty \clubpenalty
      \widowpenalty4000%
      \sfcode`\.\@m}
     {\def\@noitemerr
       {\@latex@warning{Empty `thebibliography' environment}}%
      \endlist}
\makeatother

\newcommand{\eg}{\textit{e.g.}\xspace}
\newcommand{\ie}{\textit{i.e.}\xspace}
\newcommand{\etc}{\textit{etc.}\xspace}

\newcommand{\Squirrel}{\textsc{Squirrel}\xspace}

\title{\vspace{-1em} {\bf Research engineer and postdoc positions} \\[3mm]
 {Development of \Squirrel, \\
  a proof assistant for security protocols}}
\author{}
\date{}

\begin{document}

%\pagestyle{empty}
\maketitle



\paragraph*{Summary.}
We are looking for people to join the \Squirrel project (described below)
as postdocs or research engineers, for one or more years.
Funding will be provided by the cybersecurity PEPR and possibly some
other projects.
The candidate will be an employee of CNRS and enjoy associated benefits:
unemployment, retirement and health insurance, etc.

\noindent Tentative starting date: September 1\textsuperscript{st}, 2022.

\noindent Monthly gross salary: 2\,100 \euro $ \,-\,$ 3\,000 \euro, depending on the experience of the candidate.

\paragraph{Location.}
This announcement primarily concerns job positions at IRISA, Rennes.
Similar positions might be available at Inria Paris or LMF / Université
Paris-Saclay. This may be discussed with the contacts given below,
but you may also contact directly Adrien Koutsos \url{adrien.koutsos@inria.fr}
or Caroline Fontaine \url{caroline.fontaine@lsv.fr}.

\paragraph*{Contact.} 
Please send your applications including
CV, motivation letter, and references to Stéphanie Delaune and
David Baelde by email: \url{stephanie.delaune@irisa.fr},
\url{david.baelde@irisa.fr}.





\paragraph*{Context.}

Security protocols are distributed programs that aim at securing all
kinds of communications, by using cryptography to ensure security
properties, such as confidentiality, authentication or anonymity. Such
protocols are widely deployed, e.g. for electronic commerce on the
Internet, in banking networks, mobile phones and more recently online
elections. The goal, \ie, providing security guarantees even when
communicating over an untrusted network such as internet,
is extremely difficult to achieve.
Formal methods have proved to be a very useful tool to detect errors,
and verify the correctness of security protocols. Traditionally two
approaches have been used: the computational one, which captures
strong notions of security and offers guarantees against all
probabilistic polynomial-time attackers, and the symbolic one in which
things are modelled more 
abstractly and which is more amenable to automation. To get an idea of
this research area (at least on the symbolic approach), the interested reader can consult~\cite{FormalSecurityIOSPress}.

A few years ago, Bana and Comon proposed a new approach to security
proofs~\cite{Bana:2014:CCS}. 
This approach, which they call computationally complete symbolic
attacker (CCSA), uses the symbolic formal setting of first-order
logic, but avoids the limitations of the symbolic model. This approach
has been demonstrated on various protocols to obtain formal proofs of
security, e.g.~\cite{ComonK17}. Until recently, these proofs were only
pen-and-paper 
formal proofs, limiting the scalability and trustworthiness of the CCSA approach.

Recently, a meta-logic over the CCSA logic has been developped~\cite{BDJKS-sp21}
and implemented as part of a new proof assistant: \Squirrel\footnote{
  See \url{https://squirrel-prover.github.io/}.}.
This work has brought the first mechanized proofs of security protocols using 
the CCSA methodology. The approach is subject to active research since
then, and has notably been extended to support post-quantum security
\cite{PQBC} and protocols with states~\cite{CSF22}.

The \Squirrel proof assistant prover takes as input protocol specifications 
written
in a dialect of the applied pi-caclulus. It allows users to specify
reachability and equivalence properties (encoding security and privacy
requirements of the protocol) and to prove them using tactics. The
prover features basic automated reasoning capabilities, in an attempt
to leave to the user only the high-level aspects of the proof. The
\Squirrel prover is written in OCaml and weighs about 38k lines of
code. It currently does not rely on external tools, but is integrated
with Proof General for interactive proof development in Emacs. The
development of \Squirrel takes place on % Inria's 
Gitlab and Github, and makes intensive use of testing and continuous integration.


\paragraph*{Mission.}

The main objectives will be to contribute to the
development of the {\Squirrel} prover. This could be on practical
and/or theoretical aspects, and can be achieved in various ways 
depending on the skills and expectations of the successful applicants. The list 
below is non-exhaustive and will be discussed with the applicant.

\begin{itemize}

\item \emph{User inteface.}
  At present, the protocol and properties specification as well as the
  proofs are in text mode. We are experimenting with a web-based graphical
  output, and further web integration could be considered.

\item \emph{Executability of the specification.} A difficulty when it
  comes to analyzing a protocol is to write a correct specification
  first. To verify that the specification produced is reasonable, one
  option is to simulate protocol executions. Another possibility would be to
  translate specification between \Squirrel and other protocol verification
  tools.

\item \emph{Proof automation.}
  We have been experimenting with the use of SMT solvers to automate
  some proofs. Some native developments (e.g.\ metavariables, more powerful
  tactics, user-defined tacticals) would also be useful in this respect.

\item \emph{Case studies and extensions.}
  The job will include working \emph{with} the tool and not only \emph{on}
  its code base. This will be crucial to understand the requirements.
  Further, the application might join one of several ongoing formalization
  efforts, which include proof developments but also the extension of the
  prover with new tactics, reflecting new security assumptions or new
  reasoning techniques.

\item \emph{Maintenance and documentation.}
  Last but not least, part of the mission could be to maintain the code
  base, which has been rapidly changing over the past two years and need
  some refactoring in several places. The overall goal will be to improve
  robustness and ease future developments.
  User documentation is also lacking; automated documentation
  might be considered.

\end{itemize}

\paragraph*{Requirements.} 

An engineering degree or a master degree in computer science is required.
A PhD is \emph{not} required but of course it depends on the tasks
chosen in the non-exhaustive list above.
For tasks related to programming, we are looking for candidates with
good skills in OCaml programming. In 
particular, the ability to write, understand and debug clean, maintainable 
OCaml code is mandatory. To work on more theoretical aspects, skills
on foundations of Computer Science (logic, automated
deduction, proof assistants\ldots) will be needed.
Some knowledge in security is an asset but is not mandatory.
The knowledge of French language is not compulsory for the position.

\bibliographystyle{alpha}
\bibliography{refs-squirrel.bib}

\end{document}
